Infrastructure as Code (IaC) reframes infrastructure management as a software engineering process, where environments are declared, versioned, and reproducible \cite{morris2016iac}. This approach has become central to modern cloud operations because it reduces manual drift and improves auditability.

\subsection{IaC and Automation Foundations}
HashiCorp Terraform is widely adopted for declarative provisioning and provides a consistent execution model (\texttt{init}, \texttt{plan}, \texttt{apply}) across deployment targets \cite{hashicorp2024terraform}. Prior benchmarking work reports that orchestration overhead is typically much smaller than provider-side provisioning latency, which supports asynchronous design choices in orchestration platforms \cite{terraform_performance_2023, wjarr2024}.

\subsection{Vendor Lock-In and Standardization Literature}
Research on cloud governance emphasizes that lock-in is not only technical but also operational and organizational \cite{karamitsos2023multicloud, idc2024skills}. Standards and reference architectures (NIST, ISO/IEC, OASIS) argue for abstract interfaces and policy consistency as prerequisites for portability and control \cite{nist500292, iso17788, oasisTOSCA}.

\subsection{Gap Addressed by This Work}
While existing studies define strategic multi-cloud goals, fewer works provide implementation-level evidence showing how a unified control layer improves daily cloud operations. This paper addresses that gap with an AWS-first deployment that reports concrete synchronization cadence, health latency, provisioning flow, and cost aggregation metrics from a working system. The AWS-only focus keeps the evaluation grounded in validated functionality while preserving a path to broader cloud coverage.
