\subsection{High-Level Ecosystem}
The system architecture follows a modular, containerized approach to ensure isolation and scalability. Figure \ref{fig:ecosystem} provides a comprehensive view of the interaction between the client, the internal network, and multiple external clouds.

\begin{figure}[h]
\centering
\begin{tikzpicture}[node distance=2.2cm, auto, scale=0.8, every node/.style={scale=0.8}]
    \tikzstyle{box} = [rectangle, draw, fill=blue!5, text width=7em, text centered, rounded corners, minimum height=2.5em]
    \tikzstyle{sub} = [rectangle, draw, dashed, inner sep=10pt, fill=gray!5]
    \tikzstyle{line} = [draw, -latex']

    % Client
    \node [box] (browser) {User Browser};
    \node [above=0.2cm of browser, font=\bfseries] {Client};

    % Network Cluster
    \node [box, below=1.85cm of browser] (backend) {FastAPI Backend};
    \node [box, left=0.5cm of backend] (frontend) {React (Nginx)};
    \node [box, right=0.5cm of backend] (lb) {Reverse Proxy};
    
    \node [box, below=1.5cm of backend] (redis) {Redis Broker};
    \node [box, left=0.5cm of redis] (db) {PostgreSQL};
    \node [box, right=0.5cm of redis] (worker) {Celery Worker};

    \draw [dashed] ([xshift=-0.5cm, yshift=0.5cm]frontend.north west) rectangle ([xshift=0.5cm, yshift=-0.5cm]worker.south east);
    \node at ([yshift=0.8cm]backend.north) {\bfseries Docker Compose Network};

    % External Clouds
    \node [box, below=2.5cm of redis] (azure) {Azure Cloud};
    \node [box, left=0.5cm of azure] (aws) {AWS Cloud};
    \node [box, right=0.5cm of azure] (gcp) {GCP Cloud};
    
    \node [below=0.2cm of azure, font=\bfseries] {External Clouds};

    % Connections
    \draw [line] (browser) -- node[left] {REST} (backend);
    \draw [line] (browser) -- (frontend);
    
    \draw [line] (backend) -- (db);
    \draw [line] (backend) -- (redis);
    \draw [line] (worker) -- (redis);
    \draw [line] (worker) -| (db);
    
    \draw [line] (worker) -- (aws);
    \draw [line] (worker) -- (azure);
    \draw [line] (worker) -- (gcp);

\end{tikzpicture}
\caption{Detailed Multi-Cloud Orchestration Ecosystem}
\label{fig:ecosystem}
\end{figure}

\subsection{Component Interactions}
\begin{itemize}
    \item \textbf{FastAPI Backend}: Serves as the high-performance entry point, handling JWT authentication and resource validation.
    \item \textbf{PostgreSQL}: Stores user projects, encrypted cloud credentials, and resource state historical data.
    \item \textbf{Celery \& Redis}: Manage the asynchronous execution of Terraform tasks, ensuring the API remains responsive.
    \item \textbf{Terraform Service}: An isolated runner that dynamically injects cloud credentials into pre-defined modules to provision infrastructure.
\end{itemize}

\subsection{Workflow Design}
The provisioning workflow follows a strict sequence:
1) User selects resource through the React UI.
2) API validates and persists the request as "Pending".
3) A background task is queued in Redis.
4) The Celery worker picks up the job, retrieves encrypted credentials, and executes the Terraform plan.
5) Upon completion, the database is updated with the resource's IP/metadata, and the status changes to "Active".
