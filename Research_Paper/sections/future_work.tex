\subsection{A. Future Scope: Risk and Challenge Management}
In this project phase, risk and challenge management is framed as an explicit future-scope roadmap rather than a closed evaluation result. The roadmap emphasizes operational hardening and governance maturity.

\begin{itemize}
    \item \textbf{Terraform State Integrity}: migrate state handling to hardened remote backends (for example, S3-backed state with locking strategy), and define recovery playbooks for interrupted applies.
    \item \textbf{Credential Governance}: introduce policy-driven key rotation windows, least-privilege IAM templates, and auditable credential usage trails.
    \item \textbf{Failure Engineering}: institutionalize chaos-style drills for queue backlog, worker crash, API degradation, and partial provisioning failures.
    \item \textbf{Policy-as-Code Enforcement}: integrate pre-apply policy checks to block non-compliant configuration before resource creation.
    \item \textbf{Compliance Traceability}: add structured event logs aligned with governance and audit reporting requirements.
\end{itemize}

\subsection{B. Functional Evolution Roadmap}
The next functional expansions are planned in staged form:
\begin{itemize}
    \item \textbf{Stage 1 - Live Operational Feedback}: WebSocket-based real-time logs for synchronization and Terraform execution.
    \item \textbf{Stage 2 - Cost Intelligence}: recommendation engine for idle assets, right-sizing opportunities, and budget alerts.
    \item \textbf{Stage 3 - Predictive Governance}: anomaly scoring and risk-prioritized action queues.
    \item \textbf{Stage 4 - Controlled Multi-Cloud Extension}: reintroduce Azure and GCP modules only after AWS baseline hardening targets are consistently met.
\end{itemize}

\subsection{C. Research Extensions}
Future academic work can evaluate comparative outcomes across providers, perform longitudinal drift studies, and test decision-support effectiveness of AI-guided optimization against manual operator baselines. These directions would strengthen external validity while preserving the implementation-centered approach introduced in this paper.
