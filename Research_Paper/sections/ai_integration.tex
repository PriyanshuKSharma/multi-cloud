While the current implementation emphasizes stable provisioning and monitoring, the intelligence layer is designed to improve decision quality for AWS operations.

\subsection{Predictive Resource Sizing}
The proposed intelligence pipeline uses historical utilization signals to recommend right-sized EC2 instance classes \cite{researchgate_ml_2025}. For example, persistently low utilization on larger instances can trigger recommendations to shift toward burstable families such as \texttt{t3.medium}, reducing unnecessary compute cost \cite{cost_optim_2025}.

\subsection{Anomaly Detection in AWS Utilization}
The monitoring layer can apply unsupervised anomaly detection to AWS resource and billing traces. This helps identify unusual cost or activity spikes that may indicate over-provisioning, misconfiguration, or security-relevant misuse \cite{ijsat_ai_2024}.

\subsection{Smart AWS Cost Optimization}
Cost intelligence correlates AWS service spend with provisioned resources and tags, producing actionable summaries such as unattached volumes, idle instances, and low-value network allocations \cite{hashicorp_sentinel_2024}.

\subsection{AWS Resource Mapping and Abstraction}
Cloud Simplify maintains an abstraction layer that maps user-facing labels to AWS-native services and parameter sets. Table \ref{tab:aws_mapping} shows this mapping in the current implementation.

\begin{table}[h]
\centering
\caption{AWS Resource Mapping in Cloud Simplify}
\label{tab:aws_mapping}
\begin{tabular}{|l|l|l|}
\hline
\textbf{Intent Layer} & \textbf{Cloud Simplify Label} & \textbf{AWS Service/Shape} \\ \hline
General Compute & \texttt{standard-vm-v1} & EC2 \texttt{t3.medium} \\ \hline
Burst Workload Compute & \texttt{burst-vm-v1} & EC2 \texttt{t3.small}/\texttt{t3.medium} \\ \hline
Object Storage & \texttt{storage-standard-v1} & S3 Standard Bucket \\ \hline
Network Foundation & \texttt{network-base-v1} & VPC + Subnet + Security Group \\ \hline
\end{tabular}
\end{table}
