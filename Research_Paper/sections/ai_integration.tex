The current implementation of Cloud Simplify prioritizes deterministic provisioning and reliable telemetry collection. On top of this baseline, the project defines an "intelligence layer" intended to improve operational decisions without replacing operator control.

\subsection{Baseline First, Intelligence Second}
A common failure pattern in cloud tooling is adding predictive features before stabilizing operational data pipelines. This work takes the opposite route: synchronization, state tracking, and cost collection are treated as prerequisites for trustworthy recommendations. In this context, AI integration is positioned as a decision-support capability rather than an autonomous control loop.

\subsection{Predictive Resource Sizing for AWS Workloads}
Using historical utilization indicators, the platform can recommend right-sized EC2 instance classes and flag persistent over-provisioning patterns \cite{researchgate_ml_2025, cost_optim_2025}. For example, workloads with low sustained utilization can be mapped from larger families toward burstable tiers such as \texttt{t3.medium}, reducing excess compute spend while preserving application availability targets.

\subsection{Utilization Anomaly Detection}
Anomaly detection in AWS telemetry can identify abrupt deviations in activity and cost signatures. Such deviations may indicate configuration errors, demand spikes, unoptimized scaling policies, or potentially malicious consumption patterns \cite{ijsat_ai_2024}. In practical terms, anomaly output can be linked to alerting and investigation workflows rather than automatic destructive actions, preserving operational safety.

\subsection{Cost-Aware Recommendations}
Cost signals become more useful when correlated with resource state and tags. Cloud Simplify's planned recommendation layer identifies candidates for cleanup (idle resources, unattached components, underused instances) and surfaces ranked actions with expected financial impact \cite{hashicorp_sentinel_2024}. This framing keeps optimization actionable and auditable.

\subsection{AWS Mapping and Abstraction Layer}
A robust intelligence layer requires stable semantic mapping between user intent and provider-native resources. Table \ref{tab:aws_mapping} summarizes the current abstraction design.

\begin{table}[h]
\centering
\caption{AWS Resource Mapping in Cloud Simplify}
\label{tab:aws_mapping}
\begin{tabular}{|l|l|l|}
\hline
\textbf{Intent Layer} & \textbf{Cloud Simplify Label} & \textbf{AWS Service/Shape} \\ \hline
General Compute & \texttt{standard-vm-v1} & EC2 \texttt{t3.medium} \\ \hline
Burst Workload Compute & \texttt{burst-vm-v1} & EC2 \texttt{t3.small}/\texttt{t3.medium} \\ \hline
Object Storage & \texttt{storage-standard-v1} & S3 Standard Bucket \\ \hline
Network Foundation & \texttt{network-base-v1} & VPC + Subnet + Security Group \\ \hline
\end{tabular}
\end{table}

\subsection{Research Value of the AI Layer}
Even in its planned form, the intelligence strategy contributes to the paper's thesis: standardization is not only about provisioning APIs, but also about producing data quality sufficient for reliable optimization. The more consistent the operational substrate, the more defensible AI-guided decisions become.
